\documentclass[11pt]{article}

\usepackage{hyperref}
\usepackage{times}
\usepackage{tabularx}
\usepackage{graphicx}
\usepackage{color}

\textwidth 6.5in
\textheight 8.5in
\topmargin -0.25in
\oddsidemargin 1pt
\evensidemargin 1pt
% \parskip 0pt

% \renewcommand{\baselinestretch}{1.3}

\begin{document}

% \begin{titlepage}

  \title{Explorations of happiness: proposed research}

  \author{Shimon Edelman \\ Cornell University \\ \href{mailto:edelman@cornell.edu}{edelman@cornell.edu}
          \and
          Daniel Birman \\ Stanford University \\ \href{mailto:danbirman@gmail.com}}

  \maketitle
  
  \begin{abstract}

    On the emerging integrative computational view of the mind,
    happiness plays a central, evolutionarily sanctioned role in human
    existence. The present document briefly states the background for
    this conjecture and outlines three related lines of research,
    based, respectively, on agent-based evolutionary simulation, on
    behavioral experiments, and on an imaging study, that could
    substantiate it.

  \end{abstract}

%-----------------------------------------------------------------------%
\section{Background}

The computational framework for understanding the brain/mind holds
that minds are bundles of computational processes implemented by
embodied and physically and socially situated brains
\citep{Edelman08book}. Among its advantages is the possibility of an
integrated treatment of cognition, motivation, affect, personality,
and evolution --- traditionally, the subjects of independent branches
of psychology. The present brief report points out the more obvious of
the many connections among these topics and outlines some ways in
which their synthesis can be pursued empirically. 

The synthesis that I envisage rests on two foundations: (i) the
centrality of phenomenal experience (loosely termed
``consciousness''\footnote{I distinguish between phenomenal awareness,
  which is fundamental to experience, and higher-order consciousness
  \citep[ch.9]{Edelman08book}; cf.\ \citep{FeketeEdelman11}.}) in
human existence, and (ii) the centrality of evolution in shaping
existence \citep{Dobzhansky73}. A few of the conceptual links knitting
these concepts together are listed below:

\begin{itemize}

\item Generally, \emph{emotions} are computational shortcuts, whereby
  valuation processes motivate decisions and regulate behavior
  \citep{Rogers63,Minsky06}.

\item An analysis of the computational account of the mind
  \citep{Edelman12thop} suggests that self-valuation and the affective
  states that arise from it --- notably, \emph{happiness}\footnote{For
    a review of the psychology of happiness, including the distinction
    between hedonic and eudaimonic well-being, see
    \citep{RyanDeci01}.} --- serve as a key pressure point through
  which \emph{evolution} acts on the mind
  \citep{Geary05,NovareseEtAl09}.

\item The emergence of \emph{phenomenal awareness} may be due to the
  evolved role of \emph{motivation} in connecting subjective
  (phenomenal, or experienced) states to the objective state of
  affairs in the world \citep{Cleeremans08,GinsburgJablonka10}.

\item There exist heritable\footnote{As always, this should be taken
  to mean ``partially heritable''; cf.\ \citep{JablonkaLamb07}.}
  \emph{individual differences} with regard to the role that
  cognitive, motivational, and affective states and processes play in
  well-being \citep{ChenEtAl99DRD4,Lyubomirsky01}.

\item As a \emph{motivation} tool honed by \emph{evolution}, an
  individual's \emph{happiness}

  \begin{itemize}

  \item acts by facilitating prospection and future-thinking in the
    service of \emph{forethought} (\citealp*{Edelman12thop}; see also
    my
    \href{http://www.psychologytoday.com/blog/the-happiness-pursuit}{``Happiness
      of Pursuit'' blog} for some speculations on this topic);

  \item is sensitive to physical and social circumstances (see the
    \href{http://kybele.psych.cornell.edu/~edelman/Psych-4030/schedule.html}{reading
      list} for a seminar taught in Fall 2013 (Psych~4030) for
    discussions of happiness in the context of evolution, cognition,
    morality, socioeconomic inequality, class, and power).

  \item is partly socially constructed, in a manner and to an extent
    that depends on the individual's cultural upbringing
    \citep{UchidaOgihara12}.

\end{itemize}

\end{itemize}

\noindent
In the following sections, I outline three studies that use the
methods of evolutionary simulation, behavioral experiments, and
imaging, respectively, to explore the emergence of happiness and its
cognitive dynamics. 

%-----------------------------------------------------------------------%
\section{Evolutionary agent-based simulations}

The idea here is to use evolutionary simulation methods to seek
empirical evidence to the effect that happiness lies in the pursuit,
rather than in the achievement itself \citep{Edelman12thop}.

\noindent
The main points:

\begin{itemize}

  \item Use agent-based evolutionary simulation (as in, for instance,
    \citealp*{RogersEtAl11}).

  \item A population of minimally social foragers with sensors that
    are probabilistically diagnostic of the energy value of espied
    items.

  \item Happiness linked to motivation, defined via self-monitoring,
    on two time scales --- 
    
    \begin{itemize}

      \item a short-term performance average, corresponding to hedonic
        aspects;

      \item a long-term average (including social payoff),
        corresponding to eudaimonic aspects.

    \end{itemize}

  \item Occasional changes in the environment, both in the predictive
    value of the observed cues and also of the spatial distribution of
    resources.

  \item \textbf{The happiness hypothesis (evolutionary aspect)
    $HH_e$}: perpetually happy agents with weaker motivational drive
    (a tendency to ``rest on their laurels'') should get hit harder by
    environmental shifts.

\end{itemize}

\noindent
% thanks to Barb Finlay for the following remarks:
Given that the evolutionary value of happiness is posited to be the
promotion of forethought, the line of reasoning behind $H_h$ situates
it right on the main paradox of predictive models \citep{Clark13} as
sources of evaluation (and therefore of happiness).\footnote{I am
  indebted to Barb Finlay for this observation.} The paradox here is
this: the goal of such systems should be either nothing happening at
all, or perfect predictability. Indeed, it would seem that for most
vertebrates, ``happiness'' would lie very close to such
predictability; little benefit (albeit not none) would derive from
enterprise and exploration.

The kind of enterprise envisioned here as central to happiness would
then seem to be a very high end tweak of creatures of our type, whose
success depends on it.  But, then our happiness comes down to the
nature of the tweak. Thus, the main challenge of the present project
is to get the predicted effect without explicitly building its direct
precursor(s) into the simulated agents.

Subject to the above constraint, the questions that we may wish to
focus on are as follows: 

\begin{itemize}

\item[E1] Investigate the effect of a predisposition to happiness on
  the proportion in the population of individuals in whom happiness is
  associated with pursuit (rather than with attainment).

\item[E2] Investigate the effects of the characteristics of the up-
  and down-swings in instantaneous (hedonic) well-being on
  evolutionary performance. 

  Some relevant characteristics: (i) the time constants of the rise
  and fall-off of happiness; (ii) the ratio of peak to trough levels;
  (iii) the temporal spacing of peaks.

\item[E3] Investigate the effects of the characteristics of
  life-evaluation (eudaimonic) happiness on evolutionary
  performance. 

  Some relevant characteristics: (i) susceptibility to positive and
  negative life events; (ii) the dynamics of coupling with hedonic
  states.

\item[E4] Compare the above effects for East Asian (dialectic,
  interdependent) and for European/American (monotonic, independent)
  construal models for subjective well-being.

\end{itemize}

%-----------------------------------------------------------------------%
\section{Behavioral studies}

Given the capacity of the pursuit of happiness (or, semi-technically,
``approach-motivated fun''; \citealp*{GablePoole12}) to alter the
subjects' perception of time, we may make the following conjecture:

\begin{itemize}

  \item \textbf{The happiness hypothesis (cognitive aspect) $HH_c$}:
    prospection (forethought) influences the instantaneous hedonic
    state, making the present happiness fleeting.

\end{itemize}

\noindent
From $HH_c$, it follows that being primed with the concept of future
will diminish present well-being even further. To test this
hypothesis, we may resort to methods from the extensive literature on
affective habituation (i.e., decreased pleasure over repeated
viewings, particularly in homogeneous settings).  For instance,
\citet{LeventhalEtAl07} found robust habituation to positive visual
stimuli (photos of people playing water sports, which are considered
to be high-pleasure IAPS stimuli) over 20 repeated presentations, with
only very brief pleasure rating periods separating each trial. On the
theoretical side, this paper echoes many of the themes mentioned
above. For example, the authors conceive of affective habituation as
encouraging individuals to seek novel events and experiences rather
than bask in the glow of past or even easily accessible, present
glories. 

%-----------------------------------------------------------------------%
\section{Brain signatures of happiness}

Given that, as per the general view adopted here, happiness is central
to human nature and human existence, it is not entirely surprising
that happiness-related traits are found in human physiology even at
the level of the immune system \citep{FredricksonEtAl13}. This latter
finding, in particular, serves as the motivation for the following
hypothesis: 

\begin{itemize}

  \item \textbf{The happiness hypothesis (brain aspect) $HH_b$}:
    statistical patterns of brain activation carry signatures of the
    individual's hedonic response and eudaimonic valuation
    parameters. 

\end{itemize}

\noindent
To seek such signatures, we may use lexical stimuli for which
happiness-related norms have been obtained in the past
\citep{DoddsDanforth10,DoddsEtAl11} and subject the resulting evoked
potential maps or multi-voxel brain activation patterns to exploratory
multivariate analysis \citep{ReshefEtAl11}. In particular, brain
activity should be recorded under (i) stimulus-driven conditions, and
(ii) resting state, then correlated with the standard personality
profiles used by \citet{FredricksonEtAl13}, so as to quantify hedonic
and eudaimonic parameters and tendencies.

\subsubsection*{Acknowledgments}

I have benefited from discussing various aspects of this project with
Dan Birman, Barb Finlay, Tom Gilovich, and Peter Telesca.


%-----------------------------------------------------------------------------%

\bibliographystyle{chicago}
% \bibliographystyle{plainnat}
\bibliography{/Users/shimon/Documents/my}



\end{document}
%------------------END--------------------------------------------------------%
%=============================================================================%





